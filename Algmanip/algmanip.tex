\documentclass[11pt]{scrartcl}
\usepackage[sexy]{C:/Users/brian/Documents/handouts/evan}
\newcommand{\vii}{v_{\text{i}}}
\newcommand{\vff}{v_{\text{f}}}

\title{Algebraic Manipulations}
\author{Brian Zhang}
\date{\today}

\begin{document}
\maketitle

\section{Factorizations}
\begin{theorem}[Difference of Squares]
\end{theorem}

\begin{example}[Kinematic Equations]
For motion with constant acceleration $a$, we have
\begin{align*}
a(t) &= a, \\
v(t) &= v_0 + at, \\
x(t) &= x_0 + v_0t + \frac12 at^2,
\end{align*}
where $x_0$ and $v_0$ are the intial position and velocity at $t=0$. Show that if an object has a displacement $d$ with constant acceleration $a$, then the intial and final velocities satisfy
\[\vff^2 = \vii^2 + 2ad.\]
\end{example}

\begin{soln}
Note that we can write
\begin{align*}
\vff^2 &= \vii^2 + 2ad \\ 
\implies \vff^2 - \vii^2 &= 2ad \\
\implies (\vff + \vii)(\vff - \vii) &= 2ad,
\end{align*}
where we get the last line by the difference of squares theorem shown above. But since $\vff$ is equal to $\vii + at$, we have
\[(2 \vii + at)(\vii + at - \vii)  = 2ad\]
Dividing both sides by $2a$, we get
\[\vii t + \frac12 at^2 = d. \]
But this is just the third kinematic equation listed above (where we set $x(t)-x_0=d$), so we are done. 
\end{soln}

\subsection{Partial Fractions}

\section{Substitutions}

\section{Symmetry}

\section{Problems}


\end{document}